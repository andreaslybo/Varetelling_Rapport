\chapter{Vurdering}
I dette avsluttende kapittelet skal vi drøfte og vurdere vår innsats

\section{\textbf{Vurdering av arbeidsmetodikk}}
\newthought{Å bruke en variant av SCRUM} som arbeidsmetodikk har vært til stor hjelp for oss underveis i prosjektet. Siden man kan bruke SCRUM som et rammeverk og velge eller luke bort deler av metodikken kan man skreddersy arbeidsmetodikk som passer teamet. I vårt tilfelle har det å kunne inkludere oppdragsgiver underveis i prosessen via sprintpresentasjoner vært svært viktig, mens daglig standup kanskje ikke alltid har vært like viktig siden vi har jobbet så tett store deler av prosjektet. Muligheten til å raskt omstille seg til nye situasjoner er kanskje bærebjelken til smidig metodikk og som vi nevnte i (KAP 8 REFERANSE) fikk vi smake dette på alvor i den dramatiske situasjonen som utspilte seg i forbindelse med virusutbruddet av Covid-19. 

Samtidig medfører en slik arbeidsmetodikk ekstra møter og tid i forbindelse med planlegging, presentasjoner og lignende. En utfordring med dette er kanskje motivasjonen til å være forberedt og gjennomføre disse møtene. Det har derfor vært viktig for gruppen å kommunisere på en god måte slik at arbeidsmoralen forblir høy. 


