\chapter{Veien videre}

\section{\textbf{Teknisk løsning}} 
\newthought{Vi estimerer} at vi vil bruke 4 sprinter på å gjøre løsningen klar til handover, hvor Kolonial.no kan begynne implementasjon.


\textbf{2-4 sprinter:}
\\Videre ville vi gjennomført flere brukertester for å forsikre oss om at brukerne har faktisk ønsker funksjonaliteten vi utvikler. Vi ønsker f.eks. å finne en bedre løsning på å flytte varer mellom beholdningskategorier. Etter tilbakemeldinger fra brukertestene er det noe brukerne opplever som vanskelig, og vi vil derfor bruke tiden videre på å kvalitetssikre en bedre løsning på dette. Vi har heller ikke fått testet brukere over tid, altså om løsningen vår er lett nok å bruke til at bruker orker å fortsette å bruke den, eller om det blir for tungvint å vedlikeholde beholdningen sin. For å teste dette måtte vi ha satt opp brukertester som hadde gått over 1-2 uker. 

Kolonial har i dag en holdbarhetsgaranti på 7 dager, men enkelte varer har annen merking, som f.eks 3 dager holdbarhet. Vi ønsker å implementere funksjonalitet for å varsle bruker om varer som er i ferd med å gå ut på dato, samt foreslå oppskrifter til disse varene, slik at man unngår å kaste mat. Vi kan i dag hente ut holdbarhetsdato fra Kolonial.no sitt API, og implementasjonen vil derfor være rask men vi ønsker å sette av tid til brukertesting i tillegg. 

\textbf{5-7 sprinter:}
\\Etter 4 sprinter estimerer vi at vi vil være klare for en handover. Vi ville brukt resterende tid av sprintene til å skrive teknisk dokumentasjon, og beskrive sorteringsalgoritmene slik at overgangen fra vår løsning til deres egen app blir sømløs. Med tanke på at vi ikke kjenner til kodebasen til Kolonial, eller hvor mye ressurser de ville satt av til å implementere løsningen, kan vi ikke si hvor lang tid det hadde tatt utover vårt arbeid frem til handover.

\section{\textbf{Kommunikasjon}}
\newthought{Veien videre kommunikasjonsmessig} vil være å benytte prediktiv markedsføring for å sikre fremdrift i fasene og øke CLV hos hver enkelt bruker, gjennom Marketing 4.0 metoden. Her må aktiviteter og målinger fra lojalitetshjulene, kobles opp mot prediktive analysemodeller, for å sikre at hver bruker opplever kommunikasjonen som verdifull for seg og sine ønsker og behov. 

Marketing 4.0 er en tilnærming som kombinerer online og offline interaksjoner mellom bedrift og bruker, sammen med Marketing 3.0. Målet med Marketing 4.0 er å drive brukere fra kjennskap til disippel (Kotler, m.fl. 2017, 66). Marketing 3.0 handler om å identifisere brukers ønsker og behov, såkalt verdibasert markedsføring (Kotler m.fl. 2010, 39). 

Marketing 4.0 benytter maskinlæring og kunstig intelligens for å forbedre produktiviteten av markedsføringen, samtidig som det fokuserer på det menneskelige for å styrke brukerengasjement (Kotler m.fl. 2017, 46-47). Dette er en del av prediktiv markedsføring, hvor klynging er en av de vanligste metodene for å øke CLV, og optimalisere kundeengasjementet. 

Klynging er en automatisk og statistisk prosess basert på algoritmer. Algoritmene analyserer hundrevis av brukerdata samtidig, og gir innsikt i hva som trigger de forskjellige handlingene til bruker. Dette er en motsetning til segmentering, som baseres på én eller to faktorer. Klyngene settes sammen på mest hensiktsmessig måte for å oppnå konverteringer, basert på hvilke produkter eller tjenester det er mest sannsynlig at bruker er interessert i å kjøpe eller interagere med (Artun og Levin 2015, 24- 26). 

Brukerinteraksjonen gjennom kanalmiksen og digitalisering har gjort det mulig å samle inn enorme mengder data om brukerne, ved at de selv legger inn sine preferanser og demografi, som kan brukes til personifisering av markedsføringen. Gjennom et kundeforhold vil denne informasjonen bli mer og mer detaljert, etterhvert som bruker legger igjen flere spor (Artun og Levin 2015, 10). 

Prediktiv markedsføring bruker prediktive analysemetoder for å kunne levere relevant og personlig budskap til bruker, ved alle berøringspunkter, gjennom hele kundeforholdet (Artun og Levin 2015, 3). Med prediktiv markedsføring og analyse er det ingen grenser på hvor mange kampanjer man kan utforme. Disse kan man hele tiden endre og videreutvikle, basert på tall fra prediktive analyser, for å forbedre og personalisere innholdet og drive fremdrift i brukerfasene (Artun og Levin, 16-17).

Ved å benytte denne metoden vil Kolonial.no plassere de rette brukerne i passende klynger, hvor de kan tilpasse innholdet enda mer til mer detaljerte målgrupper. På grunn av den automatiserte, analytiske tilnærmingen vil klyngene alltid være oppdatert, slik at når brukere beveger seg videre i lojalitetshjulet vil de motta kommunikasjon basert på riktig fase. På denne måten vil brukere i mye større grad føle at de får verdi av innholdet de presenteres for, fordi det tilpasses basert på klyngene. Dette igjen bidrar til økt CLV gjennom hele lojalitetshjulet. 
