\chapter{Kommunikasjonsstrategi}
\section{\textbf{Lojalitetshjul}} 
En grafisk og funksjonell fremstilling av brukers reise fra \textit{kjennskap} (laveste rangering for hvor lojal en bruker kan være til merkevaren) til \textit{disippel} (høyeste rangering for hvor lojal en bruker kan være til merkevaren), på bakgrunn av Digital Involvement Cycle-modellene (se figur \ref{fig:digitalcycle}), for å sikre fremdrift i brukerfasene og økt Customer Lifetime Value (CLV). 

\begin{figure}[H] 
    \centering
    \includegraphics[width=\textwidth]{figures/komstrat/digitalcycle.png}
    \caption[Digital Involvement Cycle-modell]{Digital Involvement Cycle-modell. Til venstre ser man de forskjellige fasene en bruker går gjennom, forklart i mål for bruker for hver fase i inndelingen for mål. På toppen inndeling for hva som skal oppnås gjennom hver fase. Se vedlegg \ref{vedlegg:1} og \ref{vedlegg:2} for ferdig utfylt modell for hver av målgruppene.
    \label{fig:digitalcycle}}
\end{figure}

CLV beskriver hvor mye profitt du kan forvente å få ut av bruker gjennom kundeforholdet (Artun og Levin 2015, 64). Brukere kan i denne sammenheng sees på som aksjer, noen er verdt mer enn andre, og verdien kan endres gjennom kundeforholdet (Artun og Levin, 10). <<Bruker>> refererer til en person som benytter løsningen og handler hos Kolonial.no.

Hjulene (figur \ref{fig:lojalpersona1} og figur \ref{fig:lojalpersona2}) er smidig utformet for å lett tilpasse endringer gjennom brukerfasene, på bakgrunn av kontinuerlige målinger. De er bygget slik at man skal kunne hoppe inn i hvilken som helst fase, og jobbe uten å ha lest gjennom hele strategien.

\paragraph{\textbf{Login til Plandisc hvor hjulene kan sees i detalj:}} \\\url{https://create.plandisc.com/wheel/list}
\\Brukernavn: clesig16@student.westerdals.no 
\\Passord: SigneogEmilie2018

\begin{figure}[H] 
    \centering
    \includegraphics[width=\textwidth]{figures/komstrat/Lojalitetshjul1}
    \caption[Lojalitetshjul - persona 1]{Lojalitethjul for persona 1. Forklart nærmere i kapittel \ref{sec:persona1}.
    \label{fig:lojalpersona1}}
\end{figure}

\begin{figure}[H] 
    \centering
    \includegraphics[width=\textwidth]{figures/komstrat/Lojalitetshjul2}
    \caption[Lojalitetshjul - persona 2]{Lojalitethjul for persona 2. Forklart nærmere i kapittel \ref{sec:persona2}).
    \label{fig:lojalpersona2}}
\end{figure}

Den ytterste ringen er delt i 7, og beskriver fasen aktivitetene er planlagt å målrettes til. Den midterste ringen beskriver aktivitetene som er planlagt, hvor hver aktivitet er fargekodet for å lettere få oversikt over hvor ofte en aktivitet skal gjennomføres. Den innerste ringen er en oversikt over målingene som må gjøres for å aktivt drive optimalisering av innholdet. 

Aktivitetene i hver fase vil naturlig kjøre mer eller mindre parallelt, da brukere alltid vil være på forskjellige steder i involveringssyklusen, og samtidig krever spesialtilpasset innhold. Noen aktiviteter som er satt i en fase vil også kunne sees av brukere i andre faser, enda innholdet er rettet til fasen det er satt i. Fasene er ikke basert på tid i form av måneder og dager, men hvor lenge hver enkel bruker er i hver enkel fase, noe som alltid vil være relativt da tid i hver fase vil variere fra bruker til bruker.

Forbrukere blir mer og mer opptatt av hverandres meninger, og deler disse i stor grad på sosiale medier. Sammen danner de et bilde av produkt eller bedrift, som ofte er svært forskjellig fra slik bedriften mener å fremstå (Kotler, m.fl. 2017, 13). Derfor er er strategien lagt opp slik at Kolonial.no hele tiden viser sin tilstedeværelse, for å møte både nåværende og potensielle brukere, og aktivt være i dialog for å fremstå og skape det bildet de ønsker bruker skal ha.

Brukere blir overøst av kommunikasjon hver dag hele tiden, som ofte oppfattes som svært upersonlig da det går ut til mange brukere samtidig, og det krever derfor en mer personlig tilnærming om bruker skal interagere med kommunikasjonen (Artun og Levin 2015, 3 og 14). Bedrifter må bli synlig i mengden, og kommunisere på en måte som er meningsfull for bruker. For å klare dette bør bedriften kartlegge kundereisen, fra første interaksjon med bedrift/løsning, til konvertering, og videre kundeforhold, for å virkelig forstå de forskjellige stadiene. De bør fokusere på hvem som mottar hvilken kommunikasjon, når (Kotler, m.fl. 2017, 59).

Bruker skal føle seg sett og ivaretatt, og vil med det bruke mer tid og penger på merkevaren. Hvis vi gir verdi, vil vi få verdi tilbake (Artun og Levin, 3). Det er derfor viktig å ta høyde for at det alltid vil være endringer i bruksmønster, og teknologiske løsninger, slik at brukers ønsker, behov og verdier møtes. Hjulet er utformet smidig, nettopp for at man skal kunne gjøre disse justeringene underveis.

Lojalitetshjulene er basert på <<Digital Involvement Cycle>>-analysene. Disse analysene reflekterer hvordan merkevaren interagerer med målgruppene gjennom kundereisen. Brukerne interagerer med forskjellige touch points ettersom de beveger seg gjennom de ulike fasene i hjulene. Brukerne kan starte i hvilken som helst fase, avhengig av hvor godt kjent de er med Kolonial.no og løsningen fra før av (Kaufman og Horton 2015, 118-119).



\subsection{\textbf{Persona 1}}
\label{sec:persona1}
Se figur \ref{fig:persona1}

\subsubsection{\textbf{Fase 1: Opplyst - Kjennskap Top of Mind 5}}
I fase 1 skal vi fokusere på at forbrukeren skal kjenne til Kolonial.no som bedrift. Målet med fasen er at forbrukeren skal ha Kolonial.no på sin <<Top of Mind 5>>-liste av dagligvarehandelforretninger. Bruker skal vite at Kolonial.no eksisterer og at det er en dagligvarehandel på nettet.

\textbf{Facebook ads:} På Facebook vil det publiseres segmenterte annonser som inneholder budskap som omhandler Kolonial.no som bedrift og at de er grønne, som vil engasjere vår persona. I denne aktiviteten er målet  å skape en kjennskap til Kolonial.no. 

\textit{Måling:} For å måle i hvor stor grad dette skaper interesse, og dermed kjennskap, vil vi legge en sporingslenke/Facebook Pixel som en CTA som vil måle clickraten til nettbutikken. En Facebook Pixel er en HTML-kodesnutt som hjelper oss å måle. Den vil i dette tilfelle måle hvor mange som trykker seg inn på linken, og videre til nettbutikken (Faizi 2017). 

\textbf{Instagram ads:} På Instagram vil det publiseres rettede annonser via Facebook Manager (Facebooks eget annonseverktøy), da Manager innehar bedre segmenteringsmuligheter enn Instagram, kontoene må da være linket sammen gjennom Facebook. 

Disse annonsene vil ha samme tema som Facebook: introduksjon til Kolonial.no, med fokus på at de er grønne. 

\textit{Måling:} For å måle i hvor stor grad dette skaper interesse, og dermed kjennskap, vil vi legge en sporingslenke som en CTA som vil måle clickraten til hjemmesiden. 

\textbf{Nettaviser:} På nettavisene vil det være reklame både i form av banner og video, og disse vil veksle: en uke banner - en uke video. Med banner mener vi vanlige reklamebannere man får opp når man scroller på nettaviser, med video så mener vi en kort reklamesnutt foran videoene på for eksempel VGTV eller DBTV. Disse vil jobbe med å introdusere Kolonial.no og øke kjennskapen. 

Handlingen i reklamefilmen skal omhandle miljø, og hvor lett det er å minske matsvinnet i sitt eget hjem. Det kan være at personen i filmen skal til å lage pizza, men får en notification på mobilen om at melken går ut på dato snart. Personen begynner å lage grøt i stedet for. Vi vil vise hvor enkelt løsningen sier i fra. 

\textit{Måling:} For å måle dette vil vi legge en sporingslenke som en CTA som vil måle clickraten til hjemmesiden. 

\textbf{For å måle målet:} For å spesifikt måle <<Top of Mind 5>> vil vi utføre spørreundersøkelser gjennom Norstat/Gallup for å se hvor brukerne måtte befinne seg.

\subsubsection{\textbf{Fase 2: Interesse}}
I fase 2 skal vi fokusere på at forbrukeren skal bli interessert og nysgjerrig på Kolonial.no som merkevare. Målet med fasen er at forbrukeren skal ha Kolonial.no på sin <<Top of Mind 3>>-liste av dagligvarehandel. Vi vil at personaen skal vite at Kolonial.no er en miljøbevisst og grønn matvarehandel, da dette er faktorer som vil appellere til målgruppen. 

\textbf{Facebook organisk:} Her vil vi publisere segmenterte annonser som inneholder budskap som vil engasjere vår persona: matsvinn, miljø og at bedriften er grønn. I denne aktiviteten er det altså fokus på å skape interesse for Kolonial.no. Dette vil postes ut 1 gang i uka. 

For å måle i hvor stor grad dette skaper interesse vil vi legge en sporingslenke/Facebook Pixel som en Call-To-Action som vil måle clickraten til nettbutikken. Vi vil også se på engasjement i form av likes, delinger og kommentarer for å se interessen.

\textit{Måling:} For å måle dette vil vi legge en sporingslenke/Facebook Pixel som en Call-To-Action som vil måle clickraten til nettbutikken. 

\textbf{Instagram organisk:} Det vil legges ut segmenterte annonser på Instagram via Facebook Manager, da Manager innehar bedre segmenteringsmuligheter enn Instagram, kontoene må da være linket sammen via Facebook. Vi vil fokusere på innhold som omhandler matsvinn og at Kolonial.no er en grønn bedrift. Dette vil postes 1 gang i uka. 

\textit{Måling:} For å måle dette vil vi legge en sporingslenke som en CTA som vil måle clickraten til hjemmesiden. 

\textbf{Utendørs:} Vi vil ha reklameplakater på Adshells ved bysykkelstativ, i tillegg til på bysyklene. Disse reklamene skal fremme Kolonial.no som en grønn miljøbevisst matvarehandel på nett. Bysyklene er relevante da det er kollektiv transport og målgruppen vår er miljøbevisst. Miljøvennlige løsninger passer midt i blinken for vår målgruppe. Dette vil pågå hele sesongen. 

\textit{Måling:} For å spesifikt måle <<Top of Mind 3>> vil vi utføre spørreundersøkelser gjennom Norstat/Gallup for å se hvor brukerne måtte befinne seg. 

\textbf{Blogg:} Kolonial.no har allerede en eksisterende blogg som per dags dato ikke er veldig aktiv. En fin måte å spre kunnskap og å skape verdi for brukeren på er å bruke dette som en kanal. Det vil her postes innlegg med grønne tips og triks, oppskrifter, og også hvordan/hvorfor bedriften Kolonial.no er grønne. 

Eksempel: <<Hva gjør vi for å sørge for at bananene våre har en grønn utvikling>>. 

Blogginnleggene vil bli sendt ut på både Facebook og nyhetsbrev. Det vil postes 1 gang i uka. På blogginnleggene vil det også ligge en mulighet for å registrere seg på nyhetsbrev for å få mer informasjon, tips og triks og oppskrifter. 

\textit{Måling:} Når det kommer til måling vil det sees på hvor mange som er inne, hvor lenge, hvor de kommer fra. Vi vil også se på hvor mange som er registrert vs hvor mange som åpner opp nyhetsbrevet. Dette er for å måle interessen. 

\textbf{Nyhetsbrev:} Det er ingen nyhetsbrev spesifikke for denne fasen, men man vil via blogginnlegget kunne registrere seg på nyhetsbrevet.

\subsubsection{\textbf{Fase 3: Involvert}}
I fase 3 er målet at bruker skal laste ned appen, og registrere bruker. Bruker skal vite at appen ikke bare har som hensikt å selge, men også være med på å bekjempe matsvinn i hjemmet ved bruk av <<oppskrift basert på beholdning>>-funksjonen. Kolonial.no bekjemper også matsvinn innad i bedriften, og dette gjør at Kolonial.no er en bedrift som gjør handling ut av ord. 


\textbf{Facebook ads:} Det vil legges ut et innlegg som tilbyr nye bruker et velkomsttilbud på 200 kr avslag på første bestilling. Dette vil være med på å sikre at de som er interesserte laster ned og tester løsningen, og kanskje senere blir faste brukere. 

\textit{Måling:} For å måle dette vil vi legge en sporingslenke som en CTA som vil måle clickraten til nettbutikken. Vi vil også se hvor mange som har lastet ned appen, og brukt tilbudet. 

\textbf{Organisk Facebook:} Her vil vi publisere segmenterte annonser som inneholder budskap som vil engasjere vår persona: matsvinn, miljø og at bedriften er grønn. Vi vil også poste blogginnlegg via poster på Facebook. I denne aktiviteten er det altså fokus på å skape interesse for Kolonial.no. Dette vil postes ut 1 gang i uka. 

\textit{Måling:} For å måle dette vil vi legge en sporingslenke/Facebook Pixel som en CTA som vil måle clickraten til hjemmesiden. Vi vil også se på engasjement i form av likes, delinger og kommentarer for å se interessen.  

\textbf{Organisk Instagram:} Det vil legges ut segmenterte annonser på Instagram via Facebook Manager, da Manager innehar bedre segmenteringsmuligheter enn Instagram, kontoene må da være linket sammen via Facebook. Vi vil fokusere på innhold som omhandler matsvinn og at Kolonial.no er en grønn bedrift. Postes 1 gang i uka. 

\textit{Måling:} For å måle dette vil vi legge en sporingslenke som en CTA som vil måle clickraten til nettbutikken. Vi vil også se på engasjement i form av likes, delinger og kommentarer for å se interessen. 

\textbf{Nyhetsbrev:} Vi sender ut nyhetsbrev som er målrettet til persona 1. Det vil være alt fra inspirasjon til matretter, hvordan du skal bruke opp maten hjemme: <<Hva kan du bruke den utgåtte melka til?>>, oppskrifter basert på den enkeltes beholdning og oppskrifter til dette. Det er også her en del av blogginnleggene vil bli sendt ut. Mailene og blogginnleggene vil også reklamere for appen, og vise frem Middagsassistenten. Det vil bli sendt ut 2 ganger i uka. 

\textit{Måling:} For å måle målet <<Laste ned app og registrere bruker>> ser vi på nedlastningstallene for appen, og sammenligner med hvor mange som faktisk har registrert seg. 

\subsubsection{\textbf{Fase 4: Forpliktelse/konvertering}}
I fase 4 er målet at forbrukeren skal bruke appen av og til - at det skal være noe de vurderer  hver gang de handler. Bruker skal vite at appen ikke bare har som hensikt å selge, men også være med på å bekjempe matsvinn i hjemmet ved bruk av <<oppskrift basert på beholdning>>-funksjonen. Kolonial.no bekjemper også matsvinn innad i bedriften, og dette gjør at Kolonial.no er en bedrift som gjør handling ut av ord. Dette vil gi den miljøbevisste målgruppen en god grunn til å bruke Kolonial.no gjentatte ganger. 

Denne målgruppen vil se postene og blogginnleggene til fasene over, men i denne fasen vil det være fokus på notificationfunksjonen som forteller deg hvis du har en ubrukt vare i beholdning som holder på å gå ut på dato. 

\textbf{Notification:} Forbrukeren får en notification på mobil hvis det er en vare i beholdning som nærmer seg utløpsdato. For eksempel: <<Melken din holder på å gå ut på dato, hva med pannekaker?>>. Brukeren vil få et alternativ som passer til beholdningen, så langt det lar seg gjøre. Dette viser at Kolonial.no faktisk er opptatt av å minske matsvinn, leder til godt omdømme, i tillegg til at dette gagner brukeren. 

\textbf{Nyhetsbrev:} Hvis bruker ikke reagerer på notification og varen ikke er brukt innen 2 dager vil samme beskjed komme på e-post/nyhetsbrev. 

\textit{Måling:} For å måle dette vil det sees på hvor mange som åpner appen gjennom notification/e-post, om de utfører handlinger (bruker varen), eller om det gjøres nye kjøp. 

\subsubsection{\textbf{Fase 5: Lojalitet}}
I fase 5 skal det fokuseres på at forbrukeren skal bruke Kolonial.no og løsningen aktivt. Kolonial skal ligge på første plass på <<Top of Mind>>-listen, og være den foretrukne dagligvarehandleren. Bruker skal bruke middagsassistenten og beholdningsoversikten 7 av 10 ganger. 

\textbf{Facebook organisk:} I denne aktiviteten vil det fokuseres på resteoppskrifter og hvordan unngå matsvinn. Det vil postes innlegg på Facebook som oppfordrer til engasjement i form av likes, kommentarer og delinger. Det vil deles en <<resteoppskrift>>, og deretter oppfordres til at følgerne kan dele sine rester og tips til hva de kan brukes til. Det vil publiseres to innlegg i uka for denne fasen. 

\textit{Måling:} For å måle dette ser vi på engasjementet i form av likes, kommentarer og delinger. Vi vil også se på i hvor stor grad engasjementet i kommentarfeltet er relatert til det som det oppfordres til.

\textbf{Instagram organisk:} Det vil her brukes Instagram Stories for å sette fokus på matsvinn, dele andre brukeres bilder og oppskrifter. Alt vil foregå litt mer muntlig og lett her. Bildene fra brukere trenger ikke ha den beste kvaliteten for å bli delt. I tillegg til aktivitet på Stories vil det også være vanlige innlegg som omhandler matsvinn og resteoppskrifter: <<Se hva Tine gjorde med potetrestene sine!>>. Det vil være to faste dager i uka med Stories (i tillegg til ved behov, det skjer noe spesielt) og ett vanlig innlegg på Instagram i uka. 

\textit{Måling:} Dette vil måles via engasjement i form av likes, kommentarer, delinger og seere på Stories og vanlig innlegg. 

\textbf{Notification:} Forbrukeren får en notification på mobil hvis det er en vare i beholdning som nærmer seg utløpsdato. For eksempel: <<Melken din holder på å gå ut på dato, hva med pannekaker?>>. Brukeren vil få et alternativ som passer til beholdningen, så langt det lar seg gjøre. Dette viser at Kolonial.no faktisk er opptatt av å minske matsvinn, leder til godt omdømme, i tillegg til at dette gagner brukeren. 

\textit{Måling:} Dette måles ved å se på hvor mange åpner appen gjennom notification, og om det utføres handlinger (bruker varen).

\textbf{Nyhetsbrev:} Hvis bruker ikke reagerer på notification og varen ikke er brukt innen 2 dager vil samme beskjed komme på e-post/nyhetsbrev. 

For å måle dette vil det sees på hvor mange som åpner appen gjennom notification/e-post, om de utfører handlinger (bruker varen), eller om det gjøres nye kjøp, 

\textbf{\textit{For å måle målet:}} For å spesifikt måle <<Top of Mind 1>> vil vi utføre spørreundersøkelser gjennom Norstat/Gallup for å se hvor brukerne måtte befinne seg. 


\subsubsection{\textbf{Fase 6: Ambassadør}}
I fase 6 er ønsket at forbrukeren alltid bruker Kolonial.no, og foretrekker det foran alle de andre dagligvarehandlerne. Vi vil at forbrukeren skal bruke middagsassistenten og beholdningsoversikten 9 av 10 ganger. 

\textbf{Facebook organisk:} I denne aktiviteten vil det fokuseres på resteoppskrifter og hvordan unngå matsvinn. Det vil postes innlegg på Facebook som oppfordrer til engasjement i form av likes, kommentarer og delinger. Det vil deles en <<resteoppskrift>>, og deretter oppfordres til at følgerne kan dele sine rester og tips til hva de kan brukes til. Det vil publiseres to innlegg i uka for denne fasen. 

For å måle dette ser vi på engasjementet i form av likes, kommentarer og delinger. Vi vil også se på i hvor stor grad engasjementet i kommentarfeltet er relatert til det som det oppfordres til.

\textbf{Instagram ads:} Det vil, i likhet med Facebook, her fokuseres på resteoppskrifter og hvordan unngå matsvinn. Det vil deles ideer til resteoppskrifter, i tillegg til at man oppfordrer følgerne til å gjøre det samme. Det vil publiseres innlegg 1 gang i uka. 

For å måle dette ser vi på engasjementet i form av likes, kommentarer og delinger. 

\textbf{Notification:} Bruker får en notification på mobil hvis det er en vare i beholdningen som er ubrukt og som nærmer seg utløpstid. 

\textbf{Notification - SoLoMo:} SoLoMo er en forkortelse for Social Local Mobile. Ved bruk av SoLoMo kan vi tracke hvor bruker befinner seg. (Kaufman og Horton 2015, 53-55) Hvis bruker befinner seg i nærheten av et pick-up point, på samme tid som pick-up pointet har varer som snart går ut på dato, vil bruker få en notification på mobilen som gir den muligheten til å komme å kjøpe varer som snart utgår for en billigere penge. 

\textit{Måling:} For å måle dette vil vi se hvor mange som kom innom pick up-pointet i forhold til hvor mange som mottok notifications. 

\textbf{Nyhetsbrev:} Vi sender ut nyhetsbrev som er målrettet til persona 1. Det vil være alt fra inspirasjon til matretter, hvordan du skal bruke op maten hjemme: <<Hva kan du bruke den utgåtte melka til?>>, oppskrifter basert på den enkeltes beholdning og oppskrifter til dette. Det er også her en del av blogginnleggene vil bli sendt ut. 

\textit{Måling:} Hvis bruker ikke reagerer på notification og varen ikke er brukt innen 2 dager vil samme beskjed komme på e-post/nyhetsbrev. 

\subsubsection{\textbf{Fase 7: Disippel}}
I denne fasen er målet at brukeren skal føle en form for eierskap til løsningen. Brukeren skal være med på å spre budskapet, gjennom markedsføring som <<Friend of Mine>> og <<Word of Mouth>>. 


\textbf{Facebook ads:} På Facebook skal vi i denne fasen begynne å se etter engasjement og lojalitet. For å appellere mer til engasjementet om matsvinn vil det postes ideer til resteoppskrifter, og i tillegg legges ved en form for <<CTA>> som oppfordrer til aktivitet i kommentarfeltet: <<Hvilke rester må du bli kvitt i dag? Hva skal du lage med dine rester?>>. 

\textit{Måling:} For å måle dette ser vi på engasjementet i form av likes, kommentarer og delinger. Vi vil også se på i hvor stor grad engasjementet i kommentarfeltet er relatert til det som oppfordres. 

\textbf{Notification:} Bruker får en notification på mobil hvis det er en vare i beholdningen som er ubrukt og som nærmer seg utløpstid. 

\textit{Måling:} For å måle dette ser vi på hvor mange som åpnet appen via notification, og om du gjorde noen handlinger (bruker varen). 

\textbf{Notification - SoLoMo:} Ved bruk av SoLoMo kan vi tracke hvor bruker befinner seg (Kaufman og Horton 2015, 53-55). Hvis bruker befinner seg i nærheten av et pick-up point, på samme tid som pick-up pointet har varer som snart går ut på dato, vil bruker få en notification på mobilen som gir den muligheten til å komme å kjøpe varer som snart utgår for en billigere penge. 

\textit{Måling:} For å måle dette vil vi se hvor mange som kom innom pick up-pointet i forhold til hvor mange som mottok notificationen. 

\textbf{Nyhetsbrev:} Vi sender ut nyhetsbrev som er målrettet til persona 1. Det vil være alt fra inspirasjon til matretter, hvordan du skal bruke opp maten hjemme: <<Hva kan du bruke den utgåtte melka til?>>, oppskrifter basert på den enkeltes beholdning og oppskrifter til dette. Det er også her en del av nyhetsbrevene vil bli sendt ut. 

For å måle dette vil det sees på hvor mange som åpner appen gjennom e-post, om de utfører handlinger, eller om det gjøres nye kjøp.

\textbf{For brukerne som aktivt bruker Kolonial.no} og løsningen vil det iblant dukke opp en vervelink, som gir både deg og den nye brukeren du verver rabatt.

\textit{Måling:} Dette vil måles ved å se hvor mange brukerne verver, og om de fortsetter å bruke Kolonial.no etter første kjøp.


\subsection{\textbf{Persona 2}}
\label{sec:persona2}
Se figur \ref{fig:persona2}

\subsubsection{\textbf{Fase 1: Opplyst - kjennskap top of mind 5 }}
I fase 1 ligger fokus på å skape kjennskap til Kolonial.no som bedrift, og drive trafikk inn på nettsiden. Målet er å havne på topp 5 i brukers Top of Mind, dette kan gjøres ved spørreundersøkelser gjort av f.eks. Gallup eller Norstat. Innhold målrettes til personer som har ingen til lite digitale spor knyttet til Kolonial.no


\textbf{Nettaviser:} 
\texit{Bannerannonser} bruker logo og slogan <<Gjør ukeshandelen billig i din matbutikk på nett!>>, sammen med bilder av matretter, handleposer fulle av mat, og bilen ute på levering. Veksler med videoannonser hver andre uke. 
\textit{Videoannonser} med budskap om hvor enkelt og tidsbesparende det er å benytte Kolonial.no, kortversjon av TV-reklame. Veksler med bannerannonser hver andre uke. 

\textit{Måling:}
Clickrate på bannere og videoer, sporingslenke fra disse til nettsiden. Hente ut statistikk på hvor lenge bruker er på nettsiden, hvor på nettsiden bruker går, og om det registreres bruker og påbegynnes/fullføres bestilling. 

\textbf{TV:} Reklame med budskap om hvor enkelt og tidsbesparende det er å benytte Kolonial.no, og hvordan Kolonial.no er til hjelp i hverdagen, langversjon. Sendes i primetime på for persona 2, på kveldstid. 

\textit{Måling:} Målinger gjøres gjennom spørreundersøkelser, og kan også gjøres via tracking av IP-adresse på brukere som har smart TV, og samtykker til registrering av trafikk for å tilpasse innhold. Her vil man få innsikt i hvordan brukers mønster er når reklamepausen kommer, om andre enheter brukes i stedet for å se på TVen og få med seg Kolonial.nos reklame (Marvin 2017). 

\subsubsection{\textbf{Fase 2: Interesse - kunnskap top of mind 3}}
I fase 2 ligger fokus på å snu kjennskap til kunnskap om Kolonial.no. Bruker skal vite at Kolonial.no er en dagligvarehandel på nett, som bidrar til å effektivisere matvarehandlingen og spare tid i hverdagen, samtidig som de er en miljøfokusert bedrift. Målet er å havne på topp 3 i brukers Top of Mind, dette kan gjøres ved spørreundersøkelser gjort av f.eks. Gallup eller Norstat. Innhold målrettes til personer som har lite til noen digitale spor knyttet til Kolonial.no 

\textbf{Facebook:} Målretter annonser i FB Business Manager med budskap om hvor enkelt og tidsbesparende det er å benytte Kolonial.no. Veksler med Insta-ads. 

\textit{Måling:} Annonser inneholder CTA med <<bestill i dag>>, her hentes clickrate og statistikk på hvor lenge bruker er på nettsiden, hvor på nettsiden bruker går, og om det registreres bruker og påbegynnes/fullføres bestilling. 

\textbf{Instagram:} Målretter annonser i FB Business Manager, da det er bedre segmenteringsmuligheter enn direkte på Instagram, med budskap om hvor enkelt og tidsbesparende det er å benytte Kolonial.no. Veksler med FB-ads. 

\textit{Måling:} Annonser inneholder CTA med <<bestill i dag>>, her hentes clickrate og statistikk på hvor lenge bruker er på nettsiden, hvor på nettsiden bruker går, og om det registreres bruker og påbegynnes/fullføres bestilling. 

\textbf{Nettaviser:} Kjøpe annonsørplasser i artikkelseksjonen. Noe fokus på matsvinn, bruke opp rester, men hovedfokus er enkle, raske oppskrifter for å vise hvor raskt og effektivt det kan være å bruke Kolonial.no når man har familie. CTA til blogg for flere tips og triks for å forenkle hverdagen.  

\textit{Måling:} Clickrate annonsørinnhold, sporingslenke til nettsiden. Hente ut statistikk på hvor lenge bruker er på nettsiden, hvor på nettsiden bruker går, og om det registreres bruker og påbegynnes/fullføres bestilling. Subscribers til blogginnlegg målrettet til fase 3. 

\textbf{Utendørs:} Adshells 1 uke av 1 uke på, for å unngå å virke for aggressive. 
Leveringsbiler med logo kjører rundt, skaper en følelse av tilstedeværelse hos bruker.

\textit{Måling:} Gjøres via spørreundersøkelser.  

\subsubsection{\textbf{Fase 3: Involvert - laste ned appen og registrere bruker}}
I fase 3 ligger fokus på å konvertere til nedlastning av app og registrering av bruker. Målet er at bruker skal laste ned appen og registrere seg, og begynne å tenke at Middagsassistenten og appen til Kolonial.no gjør hverdagen litt enklere. Innhold målrettes til personer som har noen digitale spor knyttet til Kolonial.no. 

\textbf{Nettaviser:} Kjøpe \texit{annonsørplasser} i artikkelseksjonen. Fokus på enkle, raske oppskrifter for å vise hvor raskt og effektivt det kan være å bruke Kolonial.no når man har familie, samarbeid med Godt.no med oppskrifter for familien, barnebursdager, og tilsvarende, CTA til blogginnlegg. Veksler med videoannonse hver andre uke. 

\texit{Videoannonser} med budskap om hvor enkelt og tidsbesparende det er å benytte Kolonial.no, kortversjon av TV-reklame. Veksler med annonsørplasser hver andre uke. 

\textit{Måling:} Clickrate annonsørinnhold, sporingslenke til nettsiden. Hente ut statistikk på hvor lenge bruker er på nettsiden, hvor på nettsiden bruker går, og om det registreres bruker og påbegynnes/fullføres bestilling. 

\textbf{Nyhetsbrev/e-post:} Nyhetsbrev med målrettet innhold av formen <<trenger du hjelp til å komme i gang>> + <<mest populære oppskrifter akkurat nå>> + <<introtilbud>>, sendes ut 2-3 ganger i uken. 
Når blogginnlegg er skrevet sendes det ut i nyhetsbrev

\textit{Måling:} Hvor mange klikk inn på lenker i nyhetsbrev, hvor lenge på nyhetsbrev, hvor mange fører til påbegynt/fullført bestilling. 

\textbf{Blogg:} Blogg 1 gang i uken med tips og triks til mer effektiv hverdag, med enkle <<Gjør-det-selv>>-guider som viser hvordan man lage ulike oppskrifter steg-for-steg. Call-To-Action til å subscribe nyhetsbrev.

\textit{Måling:} Antall klikk, hvor de kommer fra. Hvordan er lesemønsteret på innlegget, hvor lenge er bruker hvor i innlegget. 

\subsubsection{\textbf{Fase 4: Forpliktelse/konvertering - bruker innimellom}}
I fase 4 ligger fokus på å konvertere nesten-bestillinger og passive brukere til bekreftede bestillinger og gjennomført et kjøp. Målet er at bruker skal bruke Kolonial.no og Middagsassistenten innimellom. Innhold målrettes til personer som har noen digitale spor knyttet til Kolonial.no. 

\textbf{Facebook:} Målretter annonser i FB Business Manager med budskap om raske, enkle og sunne oppskrifter. Veksler med Insta-ads. 

\textit{Måling:} Annonser inneholder CTA med <<bestill/bruk i dag>>, her hentes clickrate og statistikk på hvor lenge bruker er på nettsiden/appen, hvor på nettsiden/appen bruker går, og om det registreres bruker og påbegynnes/fullføres bestilling. Henter også informasjon på hvor mange oppskrifter som vises, og brukes, om beholdning reduseres.  

\textbf{Instagram:} Målretter annonser i FB Business Manager, da det er bedre segmenteringsmuligheter enn direkte på Instagram, med budskap om raske, enkle og sunne oppskrifter. Veksler med Insta-ads. 

\textit{Måling:} Annonser inneholder CTA med <<bestill/bruk i dag>>, her hentes clickrate og statistikk på hvor lenge bruker er på nettsiden/appen, hvor på nettsiden/appen bruker går, og om det registreres bruker og påbegynnes/fullføres bestilling. Henter også informasjon på hvor mange oppskrifter som vises, og brukes, om beholdning reduseres.  

\textbf{Nettaviser:} \texit{Bannerannonser} oppfordrer til <<bestill i dag>>. Veksler med annonsørinnhold hver andre uke. 

\texit{Annonsørinnhold} fokuserer på enkle, raske oppskrifter for å vise hvor raskt og effektivt det kan være å bruke Kolonial.no når man har familie, samarbeid med Godt.no med oppskrifter for familien, barnebursdager, og tilsvarende, CTA til blogginnlegg fra fase 3 og 5. Veksler med bannerannonser hver andre uke. 

\textit{Måling:} Clickrate annonsørinnhold, sporingslenke til nettsiden. Hente ut statistikk på hvor lenge bruker er på nettsiden, hvor på nettsiden bruker går, og om det registreres bruker og påbegynnes/fullføres bestilling. 

\textbf{Nyhetsbrev/e-post:} Om bruker ikke reagerer på notifications sendes samme info på e-post 2 dager senere. 

\textit{Måling:} Antall uåpnede notifications sender e-post, leder til bestillinger. Benyttes rabatt på bestilling eller levering oftest. 

\textbf{Notifications:} Sender ut notifications med budskap som <<mangler du det lille ekstra? Vi har ledig levering 14:00 i morgen>> + <<vi tror du trenger>> & rabatt om man bestiller til en gitt dag. Her fyller man opp tomme plasser på bilen til gitte dager og tider, for å utnytte kapasiteten på miljøbevisst måte. Man driver også mersalg for å trigge ny bestilling. Dersom bruker ikke reagerer på notifications sendes det ut e-post med samme budskap 2 dager senere. Veksler mellom rabatt på bestilling og levering. 
\textit{Måling:} Antall åpnet notifications, leder til bestillinger. Benyttes rabatt på bestilling eller levering oftest. 

\subsubsection{\textbf{Fase 5: Lojalitet}}
I fase 5 ligger fokus på å få bruker til å benytte Kolonial.no og Middagsassistenten 7 av 10 ganger. Målet er å havne på topp 1 i brukers Top of Mind, dette kan gjøres ved spørreundersøkelser gjort av f.eks. Gallup eller Norstat. Innhold målrettes til personer som har registrert bruker, og bruker Kolonial.no, innimellom. 

\textbf{Facebook:} Organiske innlegg postes 2 ganger i uken, hvor ett av disse sponses. Budskap er nye oppskrifter på Kolonial.no og nye produkter. 

\textit{Måling:} Engasjement på innlegg, bruk i app directed fra FB, bruker/lagrer oppskrifter og sletter varer fra beholdning. 

\textbf{Blogg:} Blogg 1-2 gang i uken med nye oppskrifter. CTA til registrere seg på nyhetsbrev. 

\textit{Måling:} Antall klikk, directed from. Hvordan er lesemønsteret på innlegget, hvor lenge er bruker hvor i innlegget. Brukte/lagrede oppskrifter. 

\textbf{Nyhetsbrev/e-post:} Om bruker ikke reagerer på notifications sendes samme info på e-post 2 dager senere. 

\textit{Måling:} Antall uåpnede notifications sender e-post, leder til bestillinger. Benyttes rabatt på bestilling eller levering oftest. 

\textbf{Notifications:} Sender ut notifications med budskap som <<mangler du det lille ekstra? Vi har ledig levering 14:00 i morgen>> + <<vi tror du trenger>> & rabatt om man bestiller til en gitt dag. Her fyller man opp tomme plasser på bilen til gitte dager og tider, for å utnytte kapasiteten på miljøbevisst måte. Man driver også mersalg for å trigge ny bestilling. Dersom bruker ikke reagerer på notifications sendes det ut e-post med samme budskap 2 dager senere. Veksler mellom rabatt på bestilling og levering. 
\textit{Måling:} Antall åpnet notifications, leder til bestillinger. Benyttes rabatt på bestilling eller levering oftest. 

\subsubsection{\textbf{Fase 6: Ambassadør}}
I fase 6 ligger fokus på å få bruker til å benytte Kolonial.no og Middagsassistenten som dagligvarehandel og inspirasjonskilde. Målet er at Kolonial.no brukes 9 av 10 ganger. Innhold målrettes til personer som har registrert bruker, og bruker Kolonial.no, ofte. 

\textbf{Facebook:} Organiske innlegg postes 1-2 ganger i uken, hvor ett av disse sponses. Her skal bruker oppfordres til å dele sine oppskrifter innenfor et tema, f.eks. <<hva er din go to oppskrift i en hektisk hverdag?>>. Sesongbaserte oppskrifter med inspirasjon postes også her.  

\texit{Måling:} Engasjement på innlegg, bruk i app directed fra FB, bruker/lagrer oppskrifter og sletter varer fra beholdning. Antall delte oppskrifter på innlegg som oppfordrer til det. 

\textbf{Nyhetsbrev/e-post:} Om bruker ikke reagerer på notifications sendes samme info på e-post 2 dager senere. 

\textit{Måling:} Antall uåpnede notifications sender e-post, leder til bestillinger. Benyttes rabatt på bestilling eller levering oftest. 

\textbf{Notifications:} Sender ut notifications med budskap som <<mangler du det lille ekstra? Vi har ledig levering 14 i morgen>> + <<vi tror du trenger>> + oppskrift basert på beholdning, & rabatt om man bestiller til en gitt dag. Her fyller man opp tomme plasser på bilen til gitte dager og tider, for å utnytte kapasiteten på miljøbevisst måte. Man driver også mersalg for å trigge ny bestilling. Dersom bruker ikke reagerer på notifications sendes det ut e-post med samme budskap 2 dager senere. Veksler mellom rabatt på bestilling og levering. 

\texit{Måling:} Antall åpnet notifications, leder til bestillinger. Benyttes rabatt på bestilling eller levering oftest. 

\subsubsection{\textbf{Fase 7: Disippel}}
I fase 7 ligger fokus på å groome ambassadørene til å fortsette å bruke Kolonial.no og Middagsassistenten fast. Målet er at bruker skal snakke varmt om Kolonial.no på eget initiativ. Innhold målrettes til personer som alltid bruker Kolonial.no og Middagsassistenten, og er aktive på Kolonial.nos sosiale medier. 

\textbf{Facebook:} Organiske innlegg postes 1-2 ganger i uken, hvor ett av disse sponses. Her skal bruker oppfordres til å dele sine oppskrifter innenfor et tema, f.eks. <<hva er din go to oppskrift i en hektisk hverdag?>>. Sesongbaserte oppskrifter med inspirasjon postes også her. Veksler med instagram. 
\texit{Måling:} Engasjement på innlegg, bruk i app directed fra FB, bruker/lagrer oppskrifter og sletter varer fra beholdning. Antall delte oppskrifter på innlegg som oppfordrer til det. 

\textbf{Instagram:} 1-2 poster i uka for å oppfordre til deling av oppskrifter: favoritter/restemat. 1-2 stories - deling av brukers bilder og oppskrifter/tips
Målinger: Engasjement på innlegg, bruk i app directed fra FB, bruker/lagrer oppskrifter og sletter varer fra beholdning. Antall delte oppskrifter på innlegg som oppfordrer til det. 

\textbf{Nyhetsbrev/e-post:} 1 gang i uken: forrige ukes mest populære oppskrifter + rabattkode for å verve venner <<gi en enkel hverdag i gave>>
1 gang i uken: nye oppskrifter + rabattkode for å verve venner <<gi en enkel hverdag i gave>>.

\texit{Måling:} Hvor mange klikk inn på lenker i nyhetsbrev, hvor lenge på nyhetsbrev, hvor mange fører til påbegynt/fullført bestilling. Antall lagrede oppskrifter, fjerner fra beholdning. Antall rabatter videresendt/brukt. 