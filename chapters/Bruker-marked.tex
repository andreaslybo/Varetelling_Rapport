\chapter{\textbf{Bruker og marked}}
\section{\textbf{Brukerundersøkelse}}
Det ble i andre sprintsyklus av prosjektet utarbeidet og sendt ut en spørreundersøkelse til administratorer hos Millums kunder. Formålet med undersøkelsen var å kartlegge hva slags enheter organisasjonene innehar, hva slags og hvilken versjon av operativsystem som brukes, hvor mange som deler på hver enkelt enhet og hvordan de ansatte gjennomfører varetellinger.

\subsection{\textbf{Forberedelser}}
Vi hadde en del antagelser i forkant av undersøkelsen som dannet grunnlaget for hva vi ønsket å spørre om. Det ville for eksempel være interessant å vite om hvor mange som deler én enhet med tanke på innloggingsystemet vårt. Det kunne være forvirrende for bruker om de alltid skulle måtte logge ut av enheten når de tar den i bruk. Bruk av operativsystem og versjon var av betydning for oss fordi vi måtte tilrettelegge for dette siden enkelte kritiske funksjonaliteter ikke er tilgjengelig på eldre enheter samt at applikasjonen kan skalere forskjellig på ulike enheter. Vi ville også vite hvor ofte og hvordan bedriften utfører en varetelling. Denne innsikten er viktig i forbindelse med flyt i applikasjonen. Avslutsningsvis fant vi ut at det kunne være fint for deltager å kunne legge inn egne kommentarer til oss dersom det var annen informasjon de mente var relevant for oss.

For å kartlegge tidsbruk og avdekke feil/mangler ba vi ansatte internt i Millum ta undersøkelsen, noe som resulterte i flere konkretiseringer og endringer i undersøkelsen.

\subsection{\textbf{Gjennomføring}}
I forkant av spørreundersøkelsen tok vi kontakt med markedsavdelingen til Millum for å få en liste over kontaktpersoner som var interessante å sende undersøkelsen til. Dette resultere i en liste over de 22 administratorene vi leverte undersøkelsen til. Mange av kundene ønsker å bidra til utvikling av arbeidsverktøyene deres og vi valgte derfor å sende ut én e-post til hver organisasjon med informasjon om undersøkelsen, tidsbruk og hva undersøkelsen innebar.

\subsection{\textbf{Resultat}}
Vi lot undersøkelsen stå åpen i to uker så organisasjonene hadde mulighet til å videresende undersøkelsen internt til de ulike brukerstedene. Vi fikk derfor 64 svar fra de 22 organisasjonene vi kontaktet.

De viktigste innsiktene vi fikk i denne fasen gikk på andelen iOS/Apple-brukere og Android-brukere og hvordan de i dag gjennomfører varetellinger. Mange av tilbakemeldingene vi fikk dreide seg om muligheten til å kunne lese av strekkoden på varene på lageret og telle disse. Vi kunne derfor vite at vi traff bra med de tekniske valgene vi har gjort, siden vi utviklet samtlige ønsker som ble spilt inn. 


\section{\textbf{Konkurrenter}}
Det finnes flere direkte konkurrenter til en varetellingsapplikasjon, men siden applikasjonen er integrert i en handelsløsning som Millums eksisterende kunder bruker kan man anta at de færreste av disse er direkte konkurranse for Millum. 

Grossisten ASKO leverer ikke bare mat i kongeriket, men også handelsportaler og en varetellingsapplikasjon knyttet til denne. Applikasjonen deres tilbyr mye av den samme funksjonaliteten, men mangler et brukerrettet design og helhet som vi tilbyr i løsningen vår. En av funksjonalitetene som er sterkt etterspurt av Millum sine kunder er muligheten til å håndtere avvik i varetellingen. Det vil si varer som enten ikke er lagt inn i varetellingen på forhånd, varer man ikke får treff på via strekkode eller lignende. ASKO sin løsning tilbyr bare brukerne mulighet til å opprette nye varer, mens vi gir brukerne mulighet til å knytte strekkoden mot en eksisterende vare, søke i varekatalog og legge disse direkte inn i applikasjonen i tillegg til å opprette nye varer. Dette er et klart konkurransefortrinn for Millum.