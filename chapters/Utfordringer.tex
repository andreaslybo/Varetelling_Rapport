\chapter{\textbf{Utfordringer}}
\newthought {Alle utviklingsprosjekter byr på utfordringer.} I dette kapittelet skal vi dykke litt dypere i de tekniske og menneskelige utfordringene vi hadde underveis i prosjeket.

I sammenheng med virusutbruddet av Covid-19 gikk oppdragsgiver den 19. Mars ut med permitteringsvarsler for samtlige ansatte i organisasjonen. Avdelingslederen var rask på banen til å følge opp oss for å sørge for at vi fortsatt kunne gjennomføre hovedoppgaven vår i så høy grad det var mulig. Dessverre hadde dette uventede konsekvenser for oss. Vi hadde blant annet ikke lenger mulighet til å gjennomføre planlagte brukertester med kunder, og vi ble nødt til å skrinlegge noe planlagt funksjonalitet. Siden utviklingsmiljøet til oppdragsgiveren var lukket var vi nødt til å få på plass VPN (Virtual Private Network. Brukes til å koble seg på lukket nett utenifra). Dessverre var det ikke mulig å koble seg på VPN på mobiltelefon som ga oss utfordringer med å jobbe hjemmefra. Til tross for dette hadde vi allerede kommet langt på løsningen og var tilpasningsdyktige i denne turbulente tiden. Det er i slike situasjoner man reflekterer over hvor viktig en god prosess er, så man alltid har mulighet til å tilpasse seg nye situasjoner.

Som en følge av permitteringssituasjonen hos oppdragsgiver hadde vi heller ikke mulighet til å teste applikasjonen i produksjonsmiljøer, og vi måtte derfor definere leveransekravet på nytt ettersom vi ikke kan lansere en applikasjon som ikke er testet med reel data. Selv om dette ikke var problematisk for vår del skulle vi gjerne ha fulgt hele løpet fra innsiktfasen til produksjonssetting som vi opprinnelig hadde planlagt. Vi kommer tilbake til vurdering av situasjonen i kapittel 9 (LINK HER)