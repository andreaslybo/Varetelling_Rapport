\chapter{Mål}

\section{\textbf{Kortsiktige mål}}
\newthought{Med kortsiktige mål} mener vi mål som skal nås innen en periode på 1 år. Vi har tatt utgangspunkt i tre hovedområder vi skal fokusere på for de kortsiktige målene for Kolonial.no.

\textbf{Miljø og matsvinn:}
Matsvinn er en av de største miljøutfordringene vi har i Norge per dags dato, og miljøet er noe forbrukerne også fokuserer på i større grad med miljøfokuserte tv-programmer som for eksempel Planet Plast fra NRK med Line Elvsåshagen, hvor hun viser seerne hvilken innvirkning plast har på verden. Ifølge nettstedet Matvett.no kaster vi i Norge over 350 000 tonn mat hvert år, og det meste av dette kommer direkte fra forbruker (Matvett.no) Kolonial.nos kortsiktige mål i forbindelse med dette er å melde seg inn i NHOs avtale med myndighetene om reduksjon av matsvinn.

Avtalen går ut på at avtalepartene skal jobbe sammen for å få en bedre utnyttelse av råstoffer og ressurser gjennom å forebygge og redusere matsvinnet i hele matkjeden. Eksempler på bedrifter som har signert er blant annet Norgesgruppen, Reitangruppen og Nortura (NHO). Dette vil være en godt steg både for å få bistand til hvordan man skal løse dette på best mulig måte, men også for å vise forbruker at dette er noe Kolonial.no tar på alvor.

Kolonial.no har uttalt at deres mål er å ha det laveste matsvinnet i norsk dagligvarebransje, og de sikter på godt under 1 \%. De har også uttalt at de ikke er langt unna, \textbf{og det kortsiktige målet blir dermed å komme seg til 1 \% matsvinn}. 

\textbf{Omdømme:}
Fra intervjuene vi gjorde i sprintene før jul fikk vi inntrykk av at Kolonial.no er en relativt nøytral merkevare (se vedlegg \ref{vedlegg:3}, sprint 1). En del visste hvem de var og hva de gjorde, uten å egentlig ha noe spesifikt bilde av dem. Ved å jobbe med omdømme skal befolkningen få et inntrykk av hvem Kolonial.no er som merkevare, at de er miljøengasjerte og et godt valg innen dagligvare. Kolonial.no skal være det grønne valget, og primærmålgruppen skal være klar over hvilket miljøansvar Kolonial.no tar.

Det kortsiktige målet skal være at målgruppen skal ha Kolonial.no på topp 5 i sin <<Top of Mind>>-liste. Top of Mind går ut på at bruker skal ha den spesifikke merkevaren i bevisstheten sin. Hvis noen spør målgruppa om hvilke dagligvarekjeder de vet om, vil vi at Kolonial.no skal være en av de som dukker opp i bevisstheten først.

\textbf{Trafikk på nett og sosiale medier:}
De kortsiktige målene på sosiale medier omhandler det å bygge kjennskap, og det skal bistå det kortsiktige målet om et grønt omdømme. I tillegg til dette skal engasjement, i form at både reaksjoner, kommentarer og delinger øke med 20 \%. Ifølge Janniche Adolfsen fra Idium vil et innlegg med mer likes, kommentarer og delinger veie tyngre enn innlegg uten i nyhetsfeeden på Facebook (Adolfsen 2015). Dette er også en god grunn til at vi vil øke engasjementet.

\textbf{Clickraten skal øke med 10 \% fra sosiale medier til hjemmesiden.}
\\Vi har dessverre ikke tall på hva den ligger på nå, men vi ser at den mest sannsynlig kan økes ved å bruke flere <<Call-To-Actions>> i teksten på innlegget. En Call-To-Action er et klikkbart element, som for eksempel en link eller et bilde, som oppfordrer leseren til å gjøre en handling. I dette tilfellet vil handlingen være å få dem inn på hjemmesiden/nettbutikken. Dette er for å lede bruker inn i butikken, enten for å gi mer informasjon, eller for å selge. 

Eksempel: Hvis man har postet en oppskrift på Facebook, kan man med en Call-To-Action skrive <<Les oppskriften her>> - og vil dermed ha som formål at leseren skal være interessert nok til å trykke. 


Målet for engasjement skal være å øke det med 20 \%, dette gjelder både likes/reaksjoner og kommentarer. Vi ser at engasjementet er mye høyere på postene som inneholder spørsmål til følgerne. I eksempelet under har posten med spørsmål omtrent 1,3 tusen reaksjoner, mens posten uten har 23. Ved å bruke spørsmål oftere vil engasjementet økes. 

\begin{figure}[!htbp] 
    \centering
    \includegraphics[width=\textwidth]{figures/mal/facebook}
    \caption[Mål - likes]{Facebook - likes/reaksjoner og kommentarer
    \label{fig:facebook}}
\end{figure}

\section{\textbf{Langsiktige mål}}
\newthought{Med langsiktige mål} menes det mål som skal nås innen en periode på 10 år. De langsiktige målene har de samme fokusområdene som de kortsiktige målene. 

\textbf{Miljø og matsvinn:}
Målet om å bli en grønn miljøengasjert merkevare er langsiktig og kontinuerlig. Regjeringen og den norske matbransjen har en avtale om at de skal redusere matsvinnet i Norge med 50 \% innen år 2030. Dette skal Kolonial.no være med på. Det langsiktige målet på matsvinn vil for Kolonial.no være å ha så lite som 0,1 \%.

\textbf{Omdømme:}
Det kortsiktige målet baserte seg på å nå Top of Mind - \textbf{det langsiktige målet er å nå Friend of Mine}. Denne formen for markedsføring baserer seg på at brukerne skal henvende nettverket sitt til din merkevare, hvis nettverket skulle trenge hjelp (Eskedal 2014). Denne markedsføringsformen kan minne om Word of Mouth, som også baserer seg på <<muntlig personlig reklame>> fra person til person (Pihl 2018). Det skal det fokuseres på å bygge lojalitet og kunnskap. Forbrukere stoler mer på nettverket sitt, og det er derfor dette spiller en høy faktor når forbrukere skal få et personlig forhold til merkevarer. 

\textbf{Bruke hashtaggen \#kolonialno.}
\\Da forbrukerne stoler mer på venner enn reklameplakater vil Kolonial.no i større grad enn tidligere bruke hashtaggen \#kolonialno på Instagram, i tillegg til å oppfordre brukerne til å gjøre det samme. Kolonial.no vil også i blant reposte forbrukernes bilder, som vil gjøre at forbrukerne har lyst til å bruke hashtaggen. Dette er en god måte å spre Kolonial.no på Instagram.

\textbf{Et av målene er også at Kolonial.no skal ligge på topp 3 på Top of Mind.}


\textbf{Kolonial.no skal bli oppfattet som den grønneste matvarehandelen.} De skal fremstå som engasjerte, personlige, jordnære og miljøinteresserte. 

\textbf{Middagsassistenten skal sees på som et naturlig og godt hjelpemiddel i kampen mot matsvinn, og skal benyttes (i ulik grad) av 90 \% av Kolonial.nos bruker.}

\textbf{Trafikk på nett og i sosiale medier:}
I dag bruker 40 \% av Kolonial.nos brukere appen når de bestiller varer, mens 60 \% benytter seg av nettsiden. Ved å lansere den nye løsningen til Middagsassistenten er målet at 70 \% skal bruke app, og de resterende bruke nettsiden.



