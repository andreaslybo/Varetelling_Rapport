\chapter{\textbf{Designundersøkelser og brukertester}}
\textbf{Brukertester ligger i sin helhet som vedlegg: \ref{vedlegg:3}}

\newthought{Vi valgte å utføre} både kvalitative og kvantitative undersøkelser. Kvantitative undersøkelser samler data i form av tall og mengde, mens kvalitative gir oss svar på <<hva, hvordan og hvorfor>>-spørsmål. Slik får vi gått mer i dybden (Nordbø 2017, 77\nocite{nordbo:interaksjonsdesign}). 

Den kvantitative metoden vi anvendte var spørreundersøkelse. Her får vi nådd ut til mange, på kort tid (Nordbø 2017, 85\nocite{nordbo:interaksjonsdesign}). Gjennom denne metoden fikk vi spikret målgruppen, i tillegg til å teste ut ideer og hva løsningen vår skulle inneholde. 

De kvalitative metodene vi brukte var intervju og brukertesting. Intervjuene var semistrukturerte. Dette vil si at man har en intervjumal med spørsmål man tar utgangspunkt i, men man kan gå bort fra denne ved å trekke fra eller legge til spørsmål, om det er hensiktsmessig. (Nordbø 2017, 82\nocite{nordbo:interaksjonsdesign}). Intervjuene gjorde vi for å kartlegge Kolonial.nos omdømme, i tillegg til å se på intervjuobjektenes vaner i forhold til innkjøp av mat, og testing av ideene vi hadde kommet opp med.  

Vi valgte å utføre brukertestene inspirert av intervjumetode, med spesifikke oppgaver vi ville undersøke. Noe av dette kunne vært planlagt på en bedre måte og dokumentert bedre underveis, men vi følte vi fikk den innsikten vi hadde behov for. 

Vi fikk innspill på områder som fargevalg, universell utforming, plassering av elementer, valg av tekst og de ulike funksjonene. Vi testet blant annet en som er fargeblind, som ga oss innblikk i hvordan han oppfattet løsningen vår fra dette standpunktet. 