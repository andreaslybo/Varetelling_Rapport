\chapter{Målgruppe og persona}

\section{\textbf{Målgruppe}}
\newthought{For å kartlegge} hvem Middagsassistenten skal utvikles for lagde vi to målgrupper, en primærmålgruppe og en sekundærmålgruppe. Målgruppene består av ulike segmenter og kriterier som for eksempel aldersgruppe, interesser, sivilstatus og familieliv. Ut ifra disse har vi laget en persona som tilhører hver av målgruppene. En persona er en fiktiv representant av målgruppen, som hjelper bedriften å forstå brukere og deres behov. Det er lettere å lage tilpassede markedsaktiviteter, hvis man vet hvem man lager det for (Markedspartner).

\subsection{\textbf{Primærmålgruppe: <<Fremtidens familier>>}}
\newthought{Personer som er ferdig utdannet} for 1-5 år siden, i etableringsfasen. Har gjerne samboer/stabilt forhold, og tanker om å stifte familie/ha et livslangt partnerskap. De er typisk i aldersgruppen 25 til 35 år, og er middels til veldig opptatt av miljø.

\textbf{Hvorfor:} De er teknisk anlagte, da de tilhører generasjonen Millennials og oppvokst med teknologi, og lærer raskt (Serafino 2018). De bruker og har brukt apper i stor grad. Det er lettere å få nye vaner i yngre alder, da med tanke på å benytte Middagsassistenten som et verktøy i hverdagen. Når vanen er lagt til vil det være større sannsynlighet for at de bevarer denne når de eventuelt får barn, og havner i sekundærmålgruppen. De er kjøpesterke, men miljøbevisste, og vil derfor ta seg tid til å lære en app som støtter deres hjertesak, med tanke på matsvinn og miljøpositive tiltak. 

\textbf{Antall:} Det er ca 70.000 som fullfører utdanning i året (NSD). 39\% av personer i aldersgruppen 25-29 år og 37\% av personer i aldersgruppen 30-34 år lever i en eller annen form for samboerskap (SSB 2018). Vi antar dermed at det vil være en potensiell målgruppebase på omlag 60.000 personer i året, dersom vi tar høyde for at 10.000 (14\%) av de med fullført utdanning ikke faller innenfor målgruppen. 

\subsection{\textbf{Sekundærmålgruppe: <<Barnefamilier>>}}
\newthought{Personer med barn, gift/samboer/aleneforeldre.} Litt opptatt av miljø, men mer fokus på å spare tid og penger. Typisk i aldersgruppen 30 til 40 år.

\textbf{Hvorfor:} Det er ofte fritidsaktiviteter og lignende som foregår i disse hjemmene, som gjør at prosessen med matlaging må gå fort (både innkjøp og matlagingen). De vil spare tid på å få maten direkte på døren. De vil også kunne planlegge på forhånd i eget hjem, og bruke opp varer de har for å spare penger (unngår å kaste mat/penger).  

\textbf{Antall:} Det er 634.000 familier med barn i alderen 0-17 år i Norge (SSB 2018). Dette er en veldig stor masse å ta av, og tallet vil ikke gjenspeile faktiske brukere, men viser potensialet med tanke på hvor mange familier det i Norge i dag. 

\subsection{\textbf{Persona 1}}
\textbf{Fokus:} Vil spare miljøet

\begin{figure}[!h] 
    \centering
    \includegraphics[width=\textwidth]{figures/persona/persona-bilde1}
    \caption[Persona 1]{Visuell representasjon av persona 1
    \label{fig:persona1}}
\end{figure}

\textbf{Hvorfor vil Georg bruke Kolonial.no, og da spesifikt den nye løsningen?} 
\\Ved å bruke appen har Georg full oversikt over hva han har av varer, for å unngå å kaste unødig med mat. Han kan på forhånd bestemme seg for oppskrifter han ønsker å bestille, i tillegg til å kjøpe andre varer, og får oppskrifter basert på beholdningen sin slik at han får brukt opp det som er i ferd med å gå ut på dato. På denne måten redder Georg miljøet, litt etter litt.

\textbf{Budskap og tone of voice:}
\\Tone of voice er på mange måter merkevarens personlighet. Det er måten de kommuniserer på, og hvordan de vil fremstilles (Nausthaug 2013). Primærmålgruppen skal oppfatte Kolonial.no og Middagsassistenten som en nytenkende og innovativ butikk og løsning. I dialog med brukere skal de være behjelpelige og forståelsesfulle, men også uformelle og gi inntrykk av at det skal være lavterskel å ta kontakt med dem, og at Kolonial.no er det beste valget for dem. Siden de er miljøengasjerte skal de holde en seriøs tone når de kommuniserer dette, uten å virke belærende. De skal vise at de er <<på hugget>> for bruker.


\subsection{\textbf{Persona 2}}
\textbf{Fokus:} Vil spare penger og tid

\begin{figure}[!h] 
    \centering
    \includegraphics[width=\textwidth]{figures/persona/persona-bilde2}
    \caption[Persona 2]{Visuell representasjon av persona 2
    \label{fig:persona2}}
\end{figure}

\textbf{Hvorfor vil Anne bruke Kolonial.no, og da spesifikt den nye løsningen?}\newline
Ved å bruke denne appen vil Anne få oversikt over hva hun har i beholdningen, og dermed unngå bomkjøp. Hun vil også få ideer til middager basert på beholdningen, som gjør at hun sparer tid på planlegging. Alt kommer levert direkte på døra. Trenger ikke ha barn med i butikken. 

\textbf{Budskap og tone of voice:}
\\Sekundærmålgruppen skal oppfatte Kolonial.no og Middagsassistenten som en innovativ og lettvint løsning i hverdagen. I dialogen med bruker skal de være behjelpelige og forståelsesfulle, men også uformelle og gi inntrykk av at det skal være lavterskel å ta kontakt med dem for å få hjelp, og at Kolonial.no er det beste valget for dem. De skal vise at de er <<på hugget>> for bruker, og at de vil gjøre livet deres enklest mulig. 